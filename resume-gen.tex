%General file for CV
\documentclass[margin,10pt]{res}
\setlength{\textwidth}{6in}
\addtolength{\oddsidemargin}{-.475in}
\addtolength{\evensidemargin}{-1.875in}% length of text
\usepackage[colorlinks=false,urlbordercolor={1 1 1}]{hyperref}
\usepackage[T1]{fontenc}
\usepackage{color}
\usepackage[usenames,dvipsnames]{xcolor}
% \setlength{\parskip}{0pt}
% \setlength{\parsep}{0pt}
% \setlength{\headsep}{0pt}
% \setlength{\topskip}{0pt}
% \setlength{\topmargin}{0pt}
% \setlength{\topsep}{0pt}
% \setlength{\partopsep}{0pt}
%\newcommand{\subs}[1]{\normalfont\emph{\color{Black!90}#1}}
%\newcommand{\emphs}[1]{#1}
\newcommand{\R}{\textsuperscript{\textregistered}}
\renewcommand{\th}{\textsuperscript{\emph{th}}}
\newcommand{\st}{\textsuperscript{\emph{st}}}
\newcommand{\nd}{\textsuperscript{\emph{nd}}}
\newcommand{\subs}[1]{\normalfont #1}
\newcommand{\httpmg}{http://github.com/damiansra/maxwell-gems}
%\newcommand{\subs}[1]{\normalfont\color{Black!90}#1}
\newcommand{\secs}[1]{\normalsize{\section{\subs{#1}}}}
\newcommand{\UNAM}{National Autonomous University of Mexico (UNAM)}
\newcommand{\KAUST}{King Abdullah University of Sciences and Technology (KAUST)}

\begin{document}

% personal information
\moveleft.5\hoffset\centerline{\large\bf Dami\'an Pablo (San Rom\'an Alerigi)}
\moveleft.5\hoffset\vbox{\hrule width 6in height 1pt}\smallskip
\moveleft.5\hoffset\centerline{a: 4700 KAUST, P. O. Box 1850, Thuwal, 23955-6900, Kingdom of Saudi Arabia.}
\moveleft.5\hoffset\centerline{m: +966 544 700 801, e: \href{mailto:damian.sanroman@kaust.edu.sa}{damian.sanroman@kaust.edu.sa}}

%objective
%\section{Purpose} Ph. D. Activity report

% \section{Personal information}
% \normalsize{~}
% \normalsize{\section{\subs{Date of birth}} 13 August 1985
% \section{\subs{Place of birth}} Buenos Aires, Argentina
% \section{\subs{Nationality}} Argentina, Mexico
% \section{\subs{International experience}} lived in Argentina, Brazil, Uruguay, Mexico, United States, Spain, and Saudi Arabia}

\section{Education}

\normalsize{\section{\subs{Degree}}}
%\normalsize{\section{\normalfont Degree}
Licenciatura in Physics (physicist).
\normalsize{\section{\subs{Institution}}}
College of Sciences, \UNAM, Mexico City, Mexico
% {\bf Major Subjects}\\
% \begin{ncolumn}{2}
% 	\hline
%     Quantum mechanics         	&  Classical Mechanics \\
%     Thermodynamics 				&  Statistical Physics \\
%     Electrodynamics				&  Classical wave optics \\
%     Relativity 					&  Quantum optics \\
%     Intro. to hydrodynamics 	&  Advance mathematics for physics \\
% 	Real analysis 				&  Complex analysis \\
% 	Ordinary differential equations & Partial differential equations \\
% 	\hline
% \end{ncolumn}
\secs{Major subjects}
\begin{ncolumn}{3}
	\hline
    Quantum mechanics         	&  Classical Mechanics 	&  Thermodynamics \\ 
    Statistical Physics 		&  Electrodynamics		&  Classical wave optics \\
    Relativity 					&  Quantum optics 		&  Intro. to hydrodynamics \\
    Real analysis				&  Complex analysis 	&  Ordinary differential equations \\
    Partial differential equations & Intro. to differential geometry & Adv. math for physics \\
	\hline
\end{ncolumn}
\secs{Dates}
September 2003 to March 2009
\secs{GPA}
9.67/10\\

\secs{Degree} 
Master in Science
\secs{Institution} 
Electrical Engineering department, \KAUST, Thuwal, Saudi Arabia 
\secs{Major subjects}
\begin{ncolumn}{2}
	\hline
    Semiconductor optoelectronic devices	&  Semiconductor Lasers and LEDs \\ 
    Numerical methods 			&  Electromagnetics \\
    Optics 	&  Introduction to MEMS \\
    \hline
\end{ncolumn}
\secs{Dates} September 2009 to December 2010
\secs{GPA}
3.96/4

\secs{Degree}
Philosoph\ae~Doctor, Ph. D.
\secs{Institution} 
Photonics Laboratory, Electrical Engineering department, \KAUST, Thuwal, Saudi Arabia
\secs{Research topic} 
Light control, manipulation and transformation for photonic integrated circuits (PICs)
\secs{Major achievements}
\begin{itemize}
	\item Study of the transient photon and electron induced refractive index modification in chalcogenide materials
	\item Study of planar refractive index maps for light trapping and confinement (solution to the inverse scattering problem in the paraxial approximation)
	\item Study of planar refractive index maps for light transformation (solution inverse electromagnetic problem for full vectorial description)
	\item Development of numerical toolbox for the study of wave propagation in time-varying, non-linear, anisotropic and heterogeneous materials  
\end{itemize} 

\secs{Dates} January 2011 to August 2014 (expected graduation date) \\


\section{Personal skills \& qualifications}

\secs{~~}
\secs{Languages}
Spanish (proficient), English (proficient), French (mid-level).

\secs{Computer}
\emph{Programing languages}: \href{http://www.matlab.com}{Matlab\R}~(proficient), Python (mid-advanced), Fortran (mid-advanced), C (mid-level), \LaTeX~(proficient), \href{http://www.mcs.anl.gov/petsc/}{PETSc} (mid-level).

\emph{Packages \& software}: \href{http://www.matlab.com}{Matlab\R}~(proficient), \href{http://www.wolfram.com/mathematica/}{Mathematica\R}~(advanced), \href{http://www.comsol.com}{COMSOL\R}~(mid-advanced), \href{http://www.rsoftdesign.com/}{RSoft\R}~(advanced), \href{http://www.originlab.com}{OriginLab\R}~(advanced).

\emph{Relevant experience}: Good knowledge of high performance computing and parallel programming, lead developer of \href{\httpmg}{\emph{Maxwell-GEMS Project}}, Maxwell-Generalized ElectroMagnetic Solvers, a toolbox for simulation of wave propagation  in heterogeneous, anisotropic, non-linear and time-varying media: it includes: FEM, FDTD, Spectral and Paraxial methods.

Working knowledge of numerical inverse problem solutions at full wave (variational, transformation optics and adjoint problem) and ray-tracing approximation.

\secs{Technical certifications}
\emph{Certification}: Laser Safety, Laboratory Safety, Clean-room Laboratory Safety.

\emph{Trained and certified user}: Scanning Electron Microscopy (SEM), Focus Ion Beam (SEM-FIB), Pulsed Laser Deposition (PLD), electron beam deposition (e-beam), Photon and Ramman spectroscopy.

\secs{Organizational and social}
\emph{Training and certification}: resource management, strategic planning and development, teamwork, leadership.\\

\emph{Experience}: good ability to adapt to multicultural environments, communications, and logistics.

\secs{Artistic}
\emph{Training}: piano and music theory (14 years, 1990 to 2004), writing and literature (diverse workshops 2007 - 2009),  photography (2008 -- ~).


\section{Professional profile}

\secs{~~}
\secs{Fellowships and appointments (research)}
\begin{tabular}{p{3.8in} l}
%Description & Period \\
%\hline
\emph{Title}: Junior Research Fellow \newline \emph{Institution}: Advanced Optics Laboratory, College of Sciences \UNAM \newline \emph{Description}: experiments in quantum optics, development of a novel measurement strategy for concurrent triple coincidences and theory on properties of vacuum states \newline \emph{Supervisor}: Victor M. Velazquez (\href{mailto:vicvela@ciencias.unam.mx}{vicvela@ciencias.unam.mx}) \newline ~ & September 2006 to August 2009 
\end{tabular}

\begin{tabular}{p{3.8in} l}
 \emph{Title}: Research Fellow \newline \emph{Institution}: Joint Quantum Institute, Atomic and Molecular Optics Laboratory, University of Maryland at College Park \newline \emph{Description}: Development of laser control system for quantum state measurement and preparation \newline \emph{Supervisor}: Luis Orozco (\href{mailto:lorozco@umd.edu}{lorozco@umd.edu}) \newline & March 2007 to September 2007 \end{tabular}

\begin{tabular}{p{3.8in} l}
 \emph{Title}: Research Fellow \newline  \emph{Institution}: Institut de Ciencies Fotoniques (ICFO) \newline \emph{Description}: Development of a fast polarimetry system and density function measurement device for quantum states characterization \newline \emph{Supervisor}: Morgan Mitchell (\href{mailto:mmitchell@icfo.es}{mmitchell@icfo.es}) \newline & March 2008 to October 2008
\end{tabular}

\begin{tabular}{p{3.8in} l}
 \emph{Title}: Student Research Fellow \newline \emph{Institution}: KAUST Photonics Laboratory, KAUST \newline \emph{Description}: Numerical study of strategies for light trapping, confinement and transformation in planar refractive index structures. Development of a high performance library for the solution of wave propagation in super-luminal perturbations, and time-varying-heterogeneous-anisotropic-non-linear materials \newline \emph{Supervisor}: Boon S. Ooi (\href{mailto:boon.ooi@kaust.edu.sa}{boon.ooi@kaust.edu.sa}) \newline & September 2009 -  
\end{tabular}


\secs{Education and teaching}
\begin{tabular}{p{3.8in} l}
Teacher assistant: Electrodynamics II \newline College of Sciences, \UNAM \newline & February 2006 - July 2006
\end{tabular}

\begin{tabular}{p{3.8in} l}
Teacher assistant: Introduction to Statistics (AMCS 210) \newline Computer, Electrical and Mathematical Sciences and Engineering Division, \KAUST \newline & September 2011 to December 2011
\end{tabular}

\begin{tabular}{p{3.8in} l}
Teacher assistant: Introduction to Statistics (AMCS 210) \newline Computer, Electrical and Mathematical Sciences and Engineering Division, \KAUST \newline &  September 2012 to December 2012
\end{tabular}

\secs{Appointments (academic and non-academic)}

\begin{tabular}{p{3.8in} l}
 \emph{Position}: Co-founder \& developer  \newline \emph{Institution}: \href{\httpmg}{\emph{Maxwell-GEMS Project}} \newline \emph{Description}: Lead developer of Maxwell-GEMS solvers, a toolbox for the numerical study of wave propagation in novel materials. \newline \emph{Contact}: Boon S. Ooi (\href{mailto:boon.ooi@kaust.edu.sa}{boon.ooi@kaust.edu.sa}) \newline & March 2013 - 
\end{tabular}

\begin{tabular}{p{3.8in} l}
 \emph{Position}: Vice-president  \newline \emph{Institution}: KAUST Graduate Student Council \newline \emph{Description}: The Student Council at KAUST is in charge of providing channels of communication between administration, staff (academic and support) and the students. The vice-president is in charge of coordinating the actions of the Council chairs, enabling communication and overseeing their progress. \newline \emph{Contact}: Faizi Godsi (\href{mailto:faizi.godsi@kaust.edu.sa}{faizi.godsi@kaust.edu.sa}) \newline & December 2012 - 
\end{tabular}

\begin{tabular}{p{3.8in} l}
 \emph{PosiT.}: Founder \& developer \newline \emph{Institution}: \href{http://www.souvenirsheureux.com}{les souvenirs heureux} \newline \emph{Description}: Professional photography, literature experiments and blog \newline & December 2012 - 
\end{tabular}

\begin{tabular}{p{3.8in} l}
 \emph{Position}: Co-Founder \& research fellow  \newline \emph{Institution}: Estrategia y Desarrollo A.C. \newline \emph{Description}: consultant on development strategies for clusterization \newline \emph{Contact}: Gerardo San Roman M (\href{mailto:gerardosrm@itesm.mx}{gerardosrm@itesm.mx}) \newline & December 2011 - 
\end{tabular}

\begin{tabular}{p{3.8in} l}
 \emph{Position}: Co-founder \& Chair for University Relations \newline \emph{Institution}: KAUST Graduate Student Council \newline \emph{Description}: The Student Council at KAUST is in charge of providing channels of communication between administration, staff (academic and support) and the students. The chair of university relations is in charge of enabling channels of communication between the administration and the student body for the betterment of the non-academic life on campus. \newline \emph{Contact}: Faizi Godsi (\href{mailto:faizi.godsi@kaust.edu.sa}{faizi.godsi@kaust.edu.sa}) \newline & Novemeber 2009 - December 2010 
\end{tabular}

\begin{tabular}{p{3.8in} l}
 \emph{Position}: Co-founder \newline \emph{Institution}: KAUST Emergency Student Council \newline \emph{Description}: The Student organization established by the KAUST founding class to organize efforts for the betterment of the initial phase of KAUST, improve and pro-actively participate in the development of the institution. \newline \emph{Contact}: Najah Ashry (\href{mailto:najah.ahsri@kaust.edu.sa}{najah.ahsri@kaust.edu.sa}) \newline & August 2009 - November 2009 
\end{tabular}

\begin{tabular}{p{3.8in} l}
 \emph{Position}: Co-founder \& instructor \newline \emph{Institution}: The holography workshop at the Advanced Optics Laboratory, College of Sciences, \UNAM \newline \emph{Description}: The workshop studies, teaches,  and develops novel techniques for the production of holograms and wave optics.  \newline \emph{Contact}: Victor Velazquez (\href{mailto:vicvela@ciencias.unam.mx}{vicvela@ciencias.unam.mx}) \newline & September 2007 - March 2009 
\end{tabular}


\section{Publications}

\secs{Journals (peer-reviewed)}
D. P. San-Román Alerigi, A. B. Slimane, T. K. Ng, M. Alsunaidi, and B. S. Ooi, \href{http://dx.doi.org/10.1364/OE.21.008298}{\emph{A possible approach on optical analogues of gravitational attractors}, Optics Express {\bf 21}, 8298--8310, 2013}

D. P. San-Roman-Alerigi, D. H. Anjum, Y. Zhang, X. Yang, A. B. Slimane, T. K. Ng, M. Alsunaidi, and B. S. Ooi, \href{http://link.aip.org/link/?JAP/113/044116}{\emph{Electron irradiation induced reduction of the permittivity in Chalcogenide glass (As$_2$S$_3$) thin film}, Journal of Applied Physics {\bf 113}, 044116, 2013}

D. P. San-Roman-Alerigi, T. K. Ng, M. Alsunaidi, Y. Zhang, and B. S. Ooi, \href{http://dx.doi.org/10.1364/JOSAA.29.001252}{\emph{Generation of $J_0$-Bessel-Gauss beam by an heterogeneous refractive index map}, Journal of the Optical Society of America A {\bf 29}, 1252--1258, 2012}


\secs{Conferences (peer-reviewed)}

D. P. San-Roman-Alerigi, D. H. Anjum, Y. Zhang, X. Yang, A. B. Slimane, T. K. Ng, M. Hedhili, M. Alsunaidi, and B. S. Ooi, \emph{Electron-induced giant refractive index reduction in chalcogenide glass thin films}, 7\th International Conference on Materials for Advanced Technologies (ICMAT), Singapore, 2013. Abstract ID ICMAT13-A-0568

D. P. San-Roman-Alerigi, A. B. Slimane, T. K. Ng, M. Alsunaidi, and B. S. Ooi, \emph{Photonic analogies of gravitational attractors}, Saudi International Electronics, Communications and Photonics Conference (SIECPC), Saudi Arabia, 2013. Abstract ID: 189

D. P. San-Roman-Alerigi, D. H. Anjum, Y. Zhang, X. Yang, A. B. Slimane, T. K. Ng, M. Hedhili, M. Alsunaidi, and B. S. Ooi, \emph{Time depend characterization of permittivity of chalcogenide glass As$_2$S$_3$ by electron energy loss spectroscopy}, XXII International Materials Research Congress, Symposium on Electron Microscopy of Materials, Mexico 2013. Abstract ID: IMRC2013-SIM18-ABS001

A. Ben Slimane, R. Elafandy, A. Najar, D. P. San-Roman-Alerigi, D. Anjum, T. K. Ng, and B. Ooi, \emph{Observation of ultra-broadband emission from highly disordered GaN nanoparticles}, 7\th~International Conference on Materials for Advanced Technologies (ICMAT), Singapore, 2013. Abstract ID ICMAT13-A-0863

D. P. San-Roman-Alerigi, A. B. Slimane, T. K. Ng, M. Alsunaidi, and B. S. Ooi, \emph{On a pragmatic approach to optical analogues of gravitational attractors}, Photonics Global Conference (PGC), Singapore, December 2012

D. P. San-Roman-Alerigi, N. Guerrero, and G. San Roman M., \emph{Clusterization as a strategy for the economic and social development of a region: knowledge cluster}, The Competitiveness Institute (TCI) 15\th~Annual Conference, Spain, 2012.

D. P. San-Roman-Alerigi, Y. Zhang, T. K. Ng, A. B. Slimane, and B. S. Ooi, \emph{Monolithic Silicon-based Gauss to J$_0$-Bessel-Gauss Beam Converter}, Progress in Electromagnetics Research Symposium (PIERS), Kuala Lumpur, Malaysia, 2012

Y. Zhang, D. P. San-Roman-Alerigi, T. K. Ng, B. S. Ooi, \emph{Building blocks of passive optical components fabricated in pulsed laser deposited As$_2$S$_3$ chalcogenide glass film}, 6\th~International Conference on Materials for Advanced Technologies (ICMAT), Singapore, 2011

H. Al-Falih, Y. Khan, Y. Zhang, D. P. San-Roman-Alerigi, D. Cha, B. S. Ooi and T. K. Ng, \emph{Fabrication of tuning-fork based AFM and STM Tungsten probe}, HONET 2011, Saudi Arabia, 2011

Y. Zhang, Y. Gao, D. P. San-Roman-Alerigi, T. K. Ng, and B. S. Ooi, B. Chew and M. Hedhili, D. Zhao and H. Jain, \emph{Engineering of refractive index in sulfide chalcogenide glass by direct laser writing}, Photonics Global Conference (PGC), Singapore, 2010. Abstract ID: 5705967


\secs{Books \& contributions}

D. P. San-Roman-Alerigi, J. Stahl, and V. Velazquez, \emph{Holography, an introduction to the theory of interference and scattering}, UNAM University Press, 2\nd~edition (in preparation)

B. K. Muite, et al. \emph{Parallel Spectral Numerical Methods}: Chapter \emph{Maxwell's equations}

D. P. San-Roman-Alerigi, V. Velazquez, E. Moreno and M. Grether, \emph{La naturaleza cu\'antica de la luz, anticorrelaci\'on experimental} (The quantum nature of light, experimental anti-correlation), Thesis, College of Sciences, \UNAM, Mexico, 2009


\secs{Talks, seminars and presentations}

D. P. San-Roman-Alerigi, \emph{The effects of time-varying permittivity in light propagation}, Seminar Series on Quantum and Classical Optics, College of Sciences, \UNAM, Mexico 2013.

D. P. San-Román-Alerigi, M. Alsunaidi, and B. S. Ooi, \emph{On the electron induced modification of permittivity in chalcogenide glass}, KFUPM International Physics Day: Nonlinear sciences and their applications, Saudi Arabia, 2012

D. P. San-Roman-Alerigi, Y. Zhang, T. K. Ng, A. B. Slimane, and B. S. Ooi, \emph{Optical attractors and light control on photorefractive structures}, 1\st~KAUST-UCSB-NSF Workshop on Solid State Lighting, Saudi Arabia, 2012

D. P. San-Roman-Alerigi, \emph{Similarities between space-time geometries and refractive index mappings}, Seminar Series on Quantum and Classical Optics, College of Sciences, \UNAM, Mexico 2012.

D. P. San-Roman-Alerigi, A. B. Slimane, Y. Zhang, T. K. Ng, and B. S. Ooi, \emph{Monolithic Si-based optical beam converter}, KAUST-KFUPM Second International Workshop on Photonics, Saudi Arabia, 2011

D. P. San-Roman-Alerigi, T. K. Ng, and B. S. Ooi, \emph{Towards light manipulation with plasmon polaritons}, KAUST-KFUPM First International Workshop on Photonics, Saudi Arabia, 2010

D. P. San-Roman-Alerigi, L. Orozco, \emph{Long-term computer-based frequency locking of lasers}, Symposium on undergraduate research -- Division of Laser Sciences APS, United States, 2007 

D. P. San-Roman-Alerigi, L. Orozco, \emph{A computer based frequency lock}, University of Maryland REU-MRSEC Poster Presentation, United States, 2007

D. P. San-Roman-Alerigi, J. Stahl, M. Grether, E. Lopez-Moreno, and V. Velazquez, \emph{\'Optica y holograf\'ia} (Optics and holography), Feria Internacional del Libro del Palacio de Mineria, Mexico, 2006


\section{Awards \& scholarships}

2007 UMD-REU Research Scholarship,\\ 
2008 Caixa Catalunya – ICFO Research Scholarship,\\ 
2008 UNAM-PAPIIME Research Scholarship,\\
2008 KAUST Discovery Scholar,\\
2009 KAUST Provost Award,\\
2010 KAUST Academic Excellence Award.\\

% \section{Extra-curricular activities}
% Co-founder, UNAM Holography Workshop Laboratory. March 2007,\\
% Co-founder, Student Emergency Council at KAUST, September 2009, \\
% Co-founder, University Relations Chair of the KAUST  Graduate Student Council, November 2009, \\
% Co-founder,  \emph{Estrategia y Desarrollo, A.C.}, January 2011\\
% Vice-President, KAUST  Graduate Student Council, November 2012 \\ 
% Founder - \emph{les souvenirs heureux}, professional photography & experimental literature project\\


\section{Affiliations}
2009 Founding Scholar - King Abdullah University of Science and Technology\\
2009 Founding Chair - KAUST Graduate Student Council\\
2010 Member -- Institute of Electric and Electronic Engineers,\\
2011 Member -- American Physical Society,\\
2012 Member -- Optical Society of America,\\
2012 Member -- Society for Industrial and Applied Mathematics,\\
2012 Member -- Sociedad Mexicana de Física,\\
2012 Member -- The Competitiveness Institute 


\end{document}